\documentclass{article}
\usepackage{graphicx}
\usepackage{tocloft} 
\usepackage{glossaries}
\usepackage{lipsum}
\usepackage{geometry}
\usepackage{fancyhdr} 
\usepackage{amsmath}
\usepackage{biblatex}  

\addbibresource{../common/references.bib}  

\geometry{a4paper, margin= 1in}

\makeglossaries
\newglossaryentry{term}
{
    name=term,
    description={the definition of the term}
}
\newglossaryentry{math}
{
    name=mathematics,
    description={Mathematics is math is guess those don't need to match}
}
\newglossaryentry{ai}
{
    name=Artificial Intelligence (Ai),
    description={an intelligence based on a computer system}
}

\title{Software Requirement Specification (SRS)}
\author{Group 1 }
\date{November 26 2024}

\begin{document}
\maketitle  
\pagebreak

\tableofcontents
\pagebreak

\section*{Version History}
\begin{longtable}{|c|c|p{10cm}|}
\hline
\textbf{Version} & \textbf{Date} & \textbf{Description} \\ \hline
2.0 & November 26 2024 & Filled out the introduction subsections 1.1 and 1.2.  \\ \hline
1.0 & November 26 2024 & Began document and laid out the foundation.  \\ \hline
\end{longtable}
\pagebreak

\includegraphics[width=0.3\linewidth]{../logo/csula.png} 

\includegraphics[width=0.3\linewidth]{../logo/chromeai.png} 

\section{Introduction}
\subsection{Purpose}
The purpose of this project is to describe the detailed requirements and design considerations for our project: "Google Chrome Built-in AI Challenge", available at https://googlechromeai.devpost.com/ 

The project aims to leverage Chrome's built-in \Gls{ai} APIs to integrate functionalities powered by AI models such as Gemini Nano.

As for this SRS, the goal is for it to serve as a comprehensive guide for stakeholders, developers, and testers that helps them understand the requirements and design aspects of this project.

\subsection{Intended Audience}
The intended audience for this project is:
\begin{itemize}
    \item \textbf{Developers:} Understanding the requirements and technical specifications for this project is very important for them to successfully integrate the API and properly design the user interface.
    \item \textbf{Testers:} That way they can review the functional requirements and devise test plans to validate the project's features.
    \item \textbf{Stakeholders:} So that they can ensure the project aligns with the goals of the Google Chrome Built-in AI Challenge and evaluate the outlined process for feasibility and compliance.
\end{itemize}

This document assumes the audience has some knowledge of software engineering principles, Chrome Extension development, and AI technologies.

\subsection{Overview of the Software}
The project is the development of a Chrome Extension or web application that uses Chrome's built-in \Gls{ai} APIs to integrate with other AI models. Some of the key aspects of the software include:

\begin{itemize}
    \item \lipsum[1]
\end{itemize}

This project serves as a guide for how to approach AI-driven browser-based applications that use Chrome's built-in capabilities and integrate them with other models.

\section{External Interface Requirement}
\subsection{User Interface}

\subsection{Software Interface}


\section{Legal and Ethical Considerations}
\subsection{Data storage and privacy considerations}

\subsection{Possible legal or ethical issue}

\printglossaries
\end{document}