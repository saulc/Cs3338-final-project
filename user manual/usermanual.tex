\documentclass{article}
\usepackage{graphicx}
\usepackage{tocloft} 
\usepackage{glossaries}
\usepackage{lipsum}
\usepackage{geometry}
\usepackage{fancyhdr} 
\usepackage{amsmath}
\usepackage{biblatex}  

\addbibresource{../common/references.bib}  

\geometry{a4paper, margin= 1in}

\makeglossaries
\newglossaryentry{term}
{
    name=term,
    description={the definition of the term}
}
\newglossaryentry{math}
{
    name=mathematics,
    description={Mathematics is math is guess those don't need to match}
}
\newglossaryentry{ai}
{
    name=Artificial Intelligence (Ai),
    description={an intelligence based on a computer system}
}

\title{User Manual}
\author{Group 1 }
\date{November 20 2024}

\begin{document}
\maketitle  
\pagebreak

\tableofcontents
\pagebreak

\includegraphics[width=0.3\linewidth]{../logo/csula.png} 

\includegraphics[width=0.3\linewidth]{../logo/chromeai.png} 
\section{Introduction}
The \Gls{math} term tag must be used to include a glossary just like the citation/references \gls{term}. 
\subsection{Purpose}
Our project is based on a competition from Google to use Chrome's built in \Gls{ai} API's to interact with Gemini Nano or other Ai Models in a web app or Chrome browser extension. 

The competition is available at \cite{devpost}

\cite{dev}

\subsection{Title}
\lipsum[2]
 

\section{System Requirements}
 \begin{itemize}
  \item A computer running Google Chrome Web Browser
  \item A Google Account to download from the Chrome web store
  \item A Google account to setup save your projects
  \item A microphone to use voice input features
\end{itemize}

\section{User Interface}
 

\pagebreak
\printglossaries

% include references.
\printbibliography
\end{document}
